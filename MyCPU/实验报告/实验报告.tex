\documentclass[titlepage, 11pt]{article}
\usepackage[fntef]{ctexcap}
\usepackage{txfonts}
\usepackage{hyperref}
\usepackage{fancyhdr}
\usepackage{geometry}
\usepackage{graphicx}
\usepackage{multirow}
\usepackage{booktabs}
\usepackage{enumerate}
\usepackage{txfonts}
\usepackage{clrscode3e}
\CTEXsetup[name={第,节},number={\chinese{section}}]{section}
\CTEXsetup[number={\chinese{subsection}}]{subsection}
%\CTEXsetup[number={\chinese{subsubsection}}]{subsubsection}
%\setmainfont{Microsoft YaHei}
\setmainfont{Times New Roman}
\geometry{bottom=2.5cm,top=2.5cm,left=3.5cm,right=2.5cm}
\pagestyle{fancy}
\begin{document}
	\title{\vspace{-15mm}数字逻辑电路实验报告\\\vspace{15mm}\Huge\textbf{第十三次实验:简易计算机系统}\vspace{50mm}}
	\author{张铭方\\161220169\\16级计算机系5班\\\texttt{161220169@smail.nju.edu.cn}\and 许致明\\161180162\\16级计算机系5班\\\texttt{161180162@smail.nju.edu.cn}}\date{\vspace{70mm}\today}
	\maketitle
	\section{实验目的}
	本实验的目标是在Nexys 4开发板上实现一个简单的计算机系统,能够运行简单的指令,包括循环、整数计算、函数调用、递归等。这些指令使用RISC方式编写,存储在开发板的存储器(ROM)中。开发板的另一部分存储器(RAM)用来保存程序运行中所需的数据。此外,在完成后,开发板还具有一定的输出功能,程序输入通过向ROM中初始化相应的机器码实现。
	\section{实验原理}
	\subsection{计算机系统简介}
	计算机系统主要由CPU和外部设备组成。CPU是系统中最重要的部分,它负责控制系统运行和信息处理。外部设备负责和外界进行交互,使得计算机的可以接受输入,产生相应的输出。本实验需要实现一个简化的计算机系统。下面对两种基本的系统结构做简要介绍。\par 
	%\begin{enumerate}	
	%\end{enumerate}
	第一种是冯·诺伊曼结构,这种结构被现代的大多数CPU所使用。在这种结构下,处理器使用同一个存储器,经过同一个总线传输,具有以下特点:
	\begin{enumerate}
		\item 结构上由运算器、控制器、存储器和输入/输出设备组成;
		\item 存储器是按地址访问的,每个地址是唯一的;
		\item 指令和数据都是以二进制形式存储的;
		\item 指令按顺序执行,即一般指令按照存储顺序执行,程序的分支、循环由转移指令实现;
		\item 以运算器为中心,在输入输出设备与控制器之间的数据传送都途径运算器。运算器、存储器、输入输出设备的操作以及它们之间的联系都由控制器集中控制。
	\end{enumerate}\par 
	%\end{enumerate}
	第二种是哈佛结构,它使用两个独立的存储模块,分别存储指令和数据,并具有一条独立的地址总线和一条独立的数据总线,具有以下特点:
	\begin{enumerate}
		\item 每个存储块都不允许指令和数据并存,以便实现并行处理;
		\item 利用公共地址总线访问两个存储模块(程序存储模块ROM和输出存储模块RAM),公用数据总线则被用来完成程序存储模块或数据存储模块与CPU之间的数据传输;
		\item 地址总线和数据总线由程序存储器和数据存储器分时共用。
	\end{enumerate}\par 
	数字信号处理一般需要较大的运算量和较高的运算速度,为了提高数据吞吐量,在数字信号处理中大多采用哈佛结构。本实验所构建的计算机系统就采用了哈佛结构。
	\subsection{RISC CPU简介}
	\subsubsection{RISC CPU的基本特征}
	RISC,即Reduced Instruction Set Computer,是精简指令集计算机的简称,与它相对的是CISC(Complex Instruction Set Computer)。RISC架构主要具有以下特点:
	\begin{enumerate}
	\item 只包含一些使用频率较高的指令,并用这些指令的组合来实现较为复杂指令的功能;
	\item 指令长度固定,指令格式、寻址方式比CICS少;
	\item 只有加载、存储两条指令需要访问内存,其他指令都是在寄存器和寄存器或寄存器和立即数之间进行操作;
	\item CPU中包含多个通用寄存器,执行指令过程中的数据均暂存在寄存器中,提高指令的执行速度;
	\item 常常采用流水线技术,这样大部分指令可以在一个时钟周期内完成。还可以采用超标量和超流水线技术,使指令平均执行时间小于一个时钟周期;
	\item 控制器采用组合逻辑的控制方式,不使用微程序控制的方式。
	\end{enumerate}\par 
	CPU是计算机中的核心部件。RISC架构中的CPU进行信息处理时,主要进行如下两个步骤:
	\begin{enumerate}
		\item 将数据和指令(二进制串)读入到计算机的存储器中;
		\item 从第一条开始,按顺序执行程序,直至停机,结束运行。
	\end{enumerate}
	这一过程还可以用伪代码表示如下:
	\begin{codebox}
		\zi\proc{CPU-Execute}
		\li\While\const{true}
		\Do\li\proc{Fetch} instruction $Instr[PC]$
		\li \proc{Decode}  $Instr[PC]$
		\li\proc{Execute} $Instr[PC]$
		\li $PC\gets PC+1$
	\end{codebox}\par 
	为了实现这些操作,CPU至少需要具有以下功能:
	\begin{enumerate}
		\item 取指令:当程序已经在存储器(ROM)时,首先根据程序入口地址取出一条指令。需要CPU能发出正确的地址信息和产生控制读取存储器的信号;
		\item 指令译码:这一操作即需要分析出此二进制串的意义,获得
		它指示的操作内容,并且产生相应的操作控制命令;		
		\item 执行指令:根据指令译码得到的结果,产生相应的操作控制信号序列,控制运算器、存储器、输入输出设备的动作,完成这条指令的功能。其中包含对运算结果的处理(如设置标志位)以及下一条指令的地址的形成(如根据跳转指令更改PC的值)
	\end{enumerate}\par 
	总而言之,CPU做为计算机系统的核心,主要的任务就是取出指令,解释指令,然后根据得到的结果执行相应的指令。在这些过程中,可能涉及到存储器(例如内存、内部寄存器)的读取、写入,PC内容更改(自增1或按条件跳转),以及和外部设备的数据交换(接收外设传来的输入信号,将输出信号发送给外设)。同时,CPU也需要产生相应的控制信号,使得其他部件能正确的工作。
\end{document}